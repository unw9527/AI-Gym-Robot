\subsection{Problem}
In modern society, the significance of physical well-being is progressively increasing. 
When people engage in gym workouts, it is crucial for them to ensure that their movements are executed in a proper form. 
Although some trainees opt to hire a personal trainer to maintain correct posture, not everyone can afford this luxury. 
Consequently, some individuals may fail to achieve desired results when exercising alone, and potentially cause muscle and joint damage due to incorrect movements. 
According to a study published in the Journal of Strength and Conditioning Research, ``improper weightlifting technique is one of the  most common causes of injury in weightlifting" \cite{keogh2006injury}. 
Therefore, it is crucial to maintain proper form while exercising.

\subsection{Solution}
Inspired by the fact \cite{li2019real,liu2021robotics} that the role of a personal trainer can also be fulfilled by robotics, our team aim to build an automated system that acts a gym trainer. 
We plan to achieve our goal from both the hardware and software aspects.

The hardware components of the robot comprise a mobile platform, a control console, a display screen, and a speaker. 
We also plan to equip our robot with a variety of sensors, including a camera and ultrasonic radars. 
By utilizing a camera and ultrasonic radars, the robot is capable of determining the user's location and distance.
From the software perspective, we plan to adopt algorithms such as Mask R-CNN to identify human movements when people are moving, compare with the existing action models, and give an performance evaluation. 
The use of machine learning algorithms will enable the robot to provide more accurate feedback on the user's performance and suggest areas for improvement.

Overall, our solution aims to provide users with a personalized and effective gym training experience, utilizing the latest advancements in robotics and machine learning technologies. 
By taking a comprehensive approach to implementation, we hope to create a system that is not only effective but also efficient and easy to use.

\subsection{Visual Aid}

\subsection{High-Level Requirement List}
\begin{itemize}
    \item The robot should be able to track the user's location within a certain distance (2-3 meters) with high accuracy and reliability, using sensors or other appropriate technologies.
    \item The robot should be able to recognize body key points and skeleton binding with high accuracy and speed (fewer than 10 seconds), using computer vision or other appropriate technologies.
    \item The robot should be able to evaluate the user's performance of common exercising movements, such as jumping rope and squats, or do push-up counting, with high accuracy and consistency.
    \item The robot should provide feedback to the user in real-time, indicating whether the movements are being executed correctly and providing suggestions for improvement.
    \item The robot should be able to communicate with the user in a clear and understandable manner, using appropriate feedback mechanisms, such as visual cues.
    \end{itemize}
