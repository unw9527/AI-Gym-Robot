\subsection{ Ethics }
Our human identification and tracking system will address ethical concerns related to privacy and information security. The use of camera for user identification presents potential privacy issues since the camera can capture images and other data that can be used to identify individuals. It is therefore essential to ensure that any persons captured are properly anonymized. Furthermore, the storage and transmission of data collected by the camera pose information security concerns. Unsecured data transmission and storage can result in sensitive data being accessed by hackers and sold to malicious organizations, creating significant issues for us and potentially others.

The IEEE and ACM Code of Ethics both emphasize the importance of respecting individuals' privacy and protecting their personal data. According to the IEEE Code of Ethics, engineers must "protect the privacy of others" \cite{IEEEethics} and use information solely for legitimate purposes. Similarly, the ACM Code of Ethics, Section 1.6 Respect privacy, requires computing professionals to respect others' privacy and protect the confidentiality of accessed data.

To prevent ethical breaches, we will implement appropriate security measures to ensure that the camera's collected data is securely stored and transmitted. These measures will include utilizing an encrypted data transmission protocol, deleting intermediate and temporary data when applicable, and storing reconstruction results in an encrypted folder. We will also seek informed consent from individuals whose data is being collected and ensure their privacy is maintained throughout the project.
\subsection{ Safety }
In our project, safety is our top priority, and we have identified several potential safety issues that need to be addressed. One of the most significant safety concerns is the risk of causing harm to people or property if the robot is not properly controlled, leading to accidents or collisions. Additionally, the battery and power system are crucial components that must be carefully managed to prevent short-circuits or explosions.

To address safety concerns, we will adhere to all relevant government regulations, industry standards, and campus policies related to remote control cars and unmanned vehicles. We will also take appropriate safety measures, including limiting the car's speed and operating it only in controlled environments. Before mounting the power system onto the car, we will perform frequent checks to ensure that it is functioning correctly and is in a normal state. Additionally, we will implement fail-safes to prevent any potential accidents or collisions.